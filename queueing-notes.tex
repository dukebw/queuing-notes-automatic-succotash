\documentclass[a4paper, 12pt]{article}

\usepackage{cite}
\usepackage{amsmath}
\usepackage{amsthm}
\usepackage{fullpage}
\usepackage{url}
\usepackage{graphicx}
\usepackage{inputenc}
\usepackage{titling}
\usepackage{hyperref}
\usepackage{mdframed}
\usepackage[normalem]{ulem}

\newtheorem{theorem}{Theorem}

\date{\today}
\title{Queueing Theory Reading Notes} 

\author{Brendan Duke}

\begin{document}

\maketitle

\section{Definitions}

\section{Paper Summaries}

\subsection{Performance Modeling and Design of Computer Systems: Queueing
            Theory in Action\cite{Harchol-Balter:2013:PMD:2462638}}

\subsubsection{Chapter 17 Networks of Queues and Jackson Product Form}

A \textbf{Jackson Network} is a network with probabilistic routing where jobs
are served in FCFS order, and service times are exponentially distributed.
Arrivals, from inside and outside the network, are from Poisson processes.

\begin{equation}
        \textbf{P}\{\text{state is }(n_1,\dots,n_k)\} =
                \prod_{i = 1}^k \textbf{P} \{n_i \,\text{jobs at server}\, i\} =
                \prod_{i = 1}^k \rho_{i}^{n_i} (1 - \rho_i)
\end{equation}

\subsubsection{Chapter 18 Closed Networks of Queues}

\begin{mdframed}
\uline{\textbf{\textit{Solving Closed Batch Jackson Networks}}}
\begin{enumerate}
        \item Determine $\lambda_i$'s by solving rate equations and picking a
                value for $\lambda_1$, e.g. $\lambda_1 = 1$.
        \item Compute $\rho_i = \frac{\lambda_i}{\mu_i}$ for all $i$.
        \item Set $\pi_{n_1, \dots , n_k} = C'\rho_1^{n_1} \dots \rho_k^{n_k}$
        \item Use $\sum\limits_{
                        \substack{n_1, \dots, n_k \\ \text{s.t.} \sum_i n_i = N}}
                   \pi_{n_1, \dots , n_k} = 1$ to solve for $C'$.
\end{enumerate}
\end{mdframed}

\textbf{Mean Values Analysis (MVA)} is a method for recursively computing the
expected number of jobs at a server in a closed network with a total number of
jobs $M$, namely $\textbf{E}\big[N_j^{(M)}\big]$.

\begin{theorem}
        An arriving job to a server $j$ in a closed Jackson network with
        $M > 1$ total jobs witnesses a job distribution at each server
        equivalent to that server's steady-state job distribution in the same
        network with $M - 1$ total jobs. In particular, the mean number of jobs
        that the arrival sees at server $j$ is
        $\textbf{E}\Big[N_j^{(M - 1)}\Big]$.
\end{theorem}

We would like to use the convenient variable $p_j$, which is independent of $M$, defined as,

\begin{equation}
        p_j = \frac{\lambda_j^{(M)}}{\lambda^{(M)}} =
                \frac{X^{(M)} V_j}{\sum_{j = 1}^{k} X^{(M)} V_j} =
                \frac{V_j}{\sum_{j = 1}^{k} V_j}.
\end{equation}

We use $p_j$ in the following recursive definition of $\textbf{E}\Big[T_j^{(M)}\Big]$,

\begin{equation}
        \textbf{E}\Big[T_j^{(M)}\Big] =
                \frac{1}{\mu_j} + \frac{p_j\cdot \lambda^{(M - 1)} \textbf{E}\Big[T_j^{(M - 1)}\Big]}{\mu_j}.
\end{equation}

In the above equation, we can use the fact that
$\sum_{j = 1}^{k} \textbf{E} \Big[ N_j^{(M - 1)} \Big] = M - 1$ and apply
Little's Law to find $\lambda^{(M - 1)}$ as below,

\begin{equation}
        \lambda^{(M - 1)} = \frac{M - 1}{\sum_{j = 1}^k p_j \textbf{E} \Big[ T_j^{(M - 1)} \Big]}.
\end{equation}

\bibliographystyle{IEEEtran}
\bibliography{IEEEabrv,queueing-notes}

\end{document}
